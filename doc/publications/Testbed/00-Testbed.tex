\documentclass[conference]{IEEEtran}

% correct bad hyphenation here
%\hyphenation{op-tical net-works semi-conduc-tor}
\newcommand{\ENABLECOMMENTS}{}
\usepackage{listings}
\usepackage{ifpdf}
\usepackage{wrapfig}
\usepackage{url}
\usepackage{subcaption}
\ifpdf
\usepackage[pdftex]{graphicx}
\else
\usepackage{graphicx}
\fi
\newcommand{\iap}{\textit{DREMS}\ }
%\newcommand{\iapfull}{\textbf{D}istributed \textbf{S}oftware \textbf{P}latform }
\newcommand{\iapfull}{\textbf{D}istributed \textbf{RE}altime \textbf{M}anaged \textbf{S}ystem}


\usepackage{color}

\usepackage{tikz}
\usetikzlibrary{matrix,arrows,circuits.ee,circuits.ee.IEC,shapes.geometric,shapes.misc}
\ifdefined\ENABLECOMMENTS
\newcommand{\comment}[2]{{\texttt{\color{red}#1:#2}}}
\else
\newcommand{\comment}[2]{}
\fi

\begin{document}
%
% paper title
% can use linebreaks \\ within to get better formatting as desired
\title{A Testbed to Simulate and Analyze Resilient Cyber-Physical Systems}


% author names and affiliations
% use a multiple column layout for up to three different
% affiliations
\author{ Pranav Kumar, William Emfinger and Gabor Karsai \\
Institute for Software Integrated Systems,\\ Dept. of EECS, Vanderbilt University,\\
%  1025 16th Ave S, \\
Nashville, TN 37235, USA \\
Email:\{pkumar, emfinger, gabor\}@isis.vanderbilt.edu}


% make the title area
\maketitle

\begin{abstract}
%We have created a testbed for the development and testing of Cyber-Physical Systems (CPS) applications.  This testbed incorporates smart network hardware which allows for high-fidelity emulation of the system's network characteristics, and simulation systems which allow for high-fidelity simulation of the CPS, its environment, its sensors, and its actuators.  We discuss the architecture of this testbed, the types of experiments and applications which can be run on the testbed, some of the testbed's limitations, and some extensions to the testbed. 

This paper describes a testbed for development, deployment, testing and analysis of Cyber-Physical Systems (CPS) applications. The testbed incorporates smart network hardware, allowing high-fidelity emulation of CPS network characteristics, and CPS simulation environments to enable high-frequency sensor reading, actuator control and physical environmental changes. We discuss the architecture of this testbed and present the types of experiments and applications which can be run to study hardware and software fault tolerance, resilience engines, and system stability characteristics in distributed real-time embedded systems. We also describe the scalability, limitations and potential extensions to this testbed.
\end{abstract}
\section{Introduction}


CPS are hard to develop hardware/software for; because the software is coupled with the hardware, software testing and deployment may be difficult.  Many systems require rigorous testing before final deployment, but may not be able to be tested easily in the lab or in the real world without first providing the assurances that the tests produce.  For such systems, a closed-loop simulation testbed is necessary which can fully emulate the deployed system, including the physical characteristics of the nodes, the network characteristics of the systems, and the sensors and actuators which the systems used.  Furthermore, many of these systems use specialized embedded computers which have very different software and hardware support than cloud-based testing infrastructure can provide.  

use automotive industry testbed as example, this is cost effective version for other embedded systems developers and researchers


CPS are hard to develop hardware/software for; because the software is coupled with the hardware, software testing and deployment may be difficult.  Many systems require rigorous testing before final deployment, but may not be able to be tested easily in the lab or in the real world without first providing the assurances that the tests produce.  For such systems, a closed-loop simulation testbed is necessary which can fully emulate the deployed system, including the physical characteristics of the nodes, the network characteristics of the systems, and the sensors and actuators which the systems used.  Furthermore, many of these systems use specialized embedded computers which have very different software and hardware support than cloud-based testing infrastructure can provide.  

use automotive industry testbed as example, this is cost effective version for other embedded systems developers and researchers

architectural description:
require good physics integration -> distributed -> time sync -> low jitter -> accurate timing
what kind of tests can you run?
what kind of systems is this good for?

limitations:
can't do all integration testing
usb-ethernet -> limitation for physics possibly -> what scenarios
switches can be limitations depending on applications and such -> to what can they support?
processing limitations -> dependent on testbed -> hardware prototyping boards should be used for relevant hardware
how does the system/testbed scale: switches? openflow?

future work:
test/measure jitter


\section{Testbed Requirements}
\section{The RCPS Testbed}

\subsection{Architecture}
require good physics integration -> distributed -> time sync -> low jitter -> accurate timing
what kind of tests can you run?
what kind of systems is this good for?
\subsubsection{Application Network and Emulation}
\subsubsection{Physics Network}
\subsubsection{Physics sim and interface}
\subsubsection{Actual hardware}

\subsection{System Analysis}
\subsubsection{Network Analysis}
\subsubsection{Timing Analysis}

\subsection{Experiments}
\subsubsection{Orbiter}
Good for certain types of system testing but not great for doing tests across multiple domains
\subsubsection{KSP}
Though not as accurate, allows testing between multiple interacting domains.

\subsection{Limitations}
can't do all integration testing
usb-ethernet -> limitation for physics possibly -> what scenarios
switches can be limitations depending on applications and such -> to what can they support?
processing limitations -> dependent on testbed -> hardware prototyping boards should be used for relevant hardware
how does the system/testbed scale: switches? openflow?

\subsection{Future Work/Extensions}
test/measure jitter
\subsection{Overall Architecture}
\subsubsection{Application Network and Emulation}
\subsubsection{Physics Network}
\subsubsection{Physics sim and interface}
\subsubsection{Actual hardware}

\subsection{System Analysis}
\subsubsection{Network Analysis}
\subsubsection{Timing Analysis}

\subsection{Testing}
\subsubsection{Orbiter}
Good for certain types of system testing but not great for doing tests across multiple domains
\subsubsection{KSP}
Though not as accurate, allows testing between multiple interacting domains.

\subsection{Limitations}

\subsection{Extensions}


\section{Related Research}
\section{Conclusions}

% Acknowledgement
\section*{Acknowledgment}

This work was supported by ... 

\bibliographystyle{IEEEtran}
%\bibliography{f6}

\end{document}
